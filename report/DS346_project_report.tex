\documentclass{article}

\usepackage[utf8]{inputenc}
\usepackage{amsmath}
\usepackage{amsfonts}
\usepackage{graphicx}
\usepackage{geometry}
\usepackage{fancyhdr}
\usepackage{multicol}
\usepackage{hyperref}
\usepackage{titlesec}

\geometry{margin=1in}

% Modify section formatting to prevent page breaks
\titleformat{\section}{\normalfont\Large\bfseries}{}{0em}{}
\titleformat{\subsection}[runin]{\normalfont\bfseries}{}{1em}{}
\titlespacing{\section}{0pt}{*3}{*1}
\titlespacing{\subsection}{0pt}{*2}{*1}

\title{Your Paper Title}

\author{
    \fbox{%
        \begin{minipage}{0.3\textwidth}
            \centering
            Arlo Steyn\\
            Department of Computer Science\\
            University of Stellenbosch\\
            24713848\\
            \texttt{24713848@sun.ac.za}
        \end{minipage}%  
    }%
    \hfill
    \fbox{%
        \begin{minipage}{0.3\textwidth}
            \centering
            Andre van der Merwe\\
            Department of Computer Science\\
            University of Stellenbosch\\
            24923273\\
            \texttt{24923273@sun.ac.za}
        \end{minipage}%
    }%
    \hfill
    \fbox{%
        \begin{minipage}{0.3\textwidth}
            \centering
            Stephan Delport\\
            Department of Computer Science\\
            University of Stellenbosch\\
            242710083\\
            \texttt{242710083@sun.ac.za}
        \end{minipage}%
    }
}

\date{\today}

\begin{document}

\maketitle

\begin{abstract}
This is the abstract of the paper. Summarize the main points and contributions here.
\end{abstract}

\section{Introduction}
This is where you introduce the topic of your paper.

\section{Web Scraping Cross Validated}

This section of the report provides an overview of the Cross Validated website,
details about the scraped data, and the tools used in the scraping process.

\subsection{Overview of Cross Validated}

\subsection{Data Overview}

\subsection{Tools and Techniques for Data Scraping}

\section{Wrangling The Scraped Data}

\subsection{Data Cleaning}

\subsection{Text Normalisation}

\subsection{Structuring And Saving The Data}

\section{Llama Model}

\section{Train Model}

\section{UnSloth Optimisation}

\subsection{What is UnSloth Optimisation}

\subsection{How UnSloth Optimisation Works}

\subsection{Benefits of UnSloth in Our Llama-3.2 Model}

\section{Results}

\section{Ethics}

\section{Conclusion}

\section{References}
\begin{thebibliography}{9}

\bibitem{website1}
Author Name, ``Title of the Webpage,'' \textit{Website Name}, Date Accessed. [Online]. Available: \url{http://example.com}

\bibitem{website2}
Another Author, ``Another Title of the Webpage,'' \textit{Another Website Name}, Date Accessed. [Online]. Available: \url{http://another-example.com}

\end{thebibliography}

\end{document}
